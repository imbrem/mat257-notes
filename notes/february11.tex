\documentclass{article}
\usepackage[utf8]{inputenc}

\title{MAT257 Notes}
\author{Jad Elkhaleq Ghalayini}
\date{February 8 2019}

\usepackage{amsmath}
\usepackage{amssymb}
\usepackage{amsthm}
\usepackage{mathtools}
\usepackage{enumitem}
\usepackage{graphicx}
\usepackage{cancel}

\usepackage[margin=1in]{geometry}

\newtheorem{theorem}{Theorem}
\newtheorem{lemma}{Lemma}
\newtheorem{definition}{Definition}
\newtheorem*{corollary}{Corollary}
\newtheorem{exercise}{Exercise}
\newtheorem{claim}{Claim}

\DeclareMathOperator{\Int}{Int}
\DeclareMathOperator{\grad}{grad}
\DeclareMathOperator{\Div}{div}
\DeclareMathOperator{\curl}{curl}
\DeclareMathOperator{\Ker}{Ker}
\DeclareMathOperator{\Ima}{Im}
\DeclareMathOperator{\Vol}{vol}
\DeclareMathOperator{\D}{D}

\newcommand{\reals}[0]{\mathbb{R}}
\newcommand{\nats}[0]{\mathbb{N}}
\newcommand{\ints}[0]{\mathbb{Z}}
\newcommand{\rationals}[0]{\mathbb{Q}}
\newcommand{\brac}[1]{\left(#1\right)}
\newcommand{\sbrac}[1]{\left[#1\right]}
\newcommand{\mc}[1]{\mathcal{#1}}
\newcommand{\eval}[3]{\left.#3\right|_{#1}^{#2}}
\newcommand{\ip}[2]{\left\langle#1,#2\right\rangle}
\newcommand{\prt}[2]{\frac{\partial #1}{\partial #2}}
\newcommand{\mb}[1]{\mathbf{#1}}
\newcommand{\hlfspc}[0]{\mathbb{H}}
\newcommand{\loint}[0]{\operatorname{L}\int}
\newcommand{\hiint}[0]{\operatorname{U}\int}
\newcommand{\indic}[1]{\chi_{#1}}

\begin{document}

\maketitle

\section{Vector Fields and Differential Forms}
Recall the definitions of the \underline{tangent space to \(\reals^n\) at \(a\)}, \(\text{T}\reals^n_a\) or \(\reals^n_a\). We write the standard bases of \(\reals^n_a\) as
\begin{equation}
  e_{1,a},...,e_{n,a}
\end{equation}
and define the stadard inner product
\begin{equation}
  \ip{v_a}{w_a}_a = \ip{v}{w}
\end{equation}
Using the definitions from last time, we obtain a \underline{standard orientation} of \(\reals^n_a\)
\begin{equation}
  [e_{1,a},...,e_{n,a}]
\end{equation}
Recall the following definition:
\begin{definition}
  The \underline{tangent vector} to a \(\mc{C}^1\) curve \(\gamma: [0, 1] \to \reals^n\) at \(t \in [0, 1]\) is given by
  \begin{equation}
    \gamma'(t)_{\gamma(t)} = (\gamma_1'(t),...,\gamma_n'(t))_{\gamma(t)} = \sum_{i = 1}^n\gamma_i'(t)e_{i, \gamma(t)} \in \reals^n_{\gamma(t)}
  \end{equation}
  We write
  \begin{equation}
    \gamma'(t)_{\gamma(t)} = \gamma_{*t}(e_{1, t})
  \end{equation}
  where \(e_{1, t} \in \reals^1_t\), thinking of the derivative of \(\gamma\) \(\D\gamma\) inducing a mapping
  \begin{equation}
    \gamma_{*t}: \reals_t^1 \to \reals^n_{\gamma(t)}, \gamma_{*t}(ae_{1, t}) = \D\gamma(a + t)(e_1)_{\gamma(t)}
  \end{equation}
\end{definition}
Consider now a function \(\varphi : \reals^n \to \reals^p\), and consider the curve \((\varphi \circ \gamma): [0, 1] \to \reals^p\). What's the tangent vector to \(\varphi \circ \gamma\) at \(t\)? Well, by definition, it's
\begin{equation}
  (\varphi \circ \gamma)_{*t}(e_{1, t}) = \D(\varphi \circ \gamma)(t)(e_1)_{\varphi(\gamma(t))} \label{starphi}
\end{equation}
By the Chain Rule, \ref{starphi} simplifies to
\begin{equation}
  \D\varphi(\gamma(t)) \circ \D\gamma(t) (e_1)_{\varphi(\gamma(t))} = \varphi_{*\gamma(t)}(\D\gamma(t)(e_1)_{\gamma(t)}) = \varphi_{*\gamma(t)} \circ \gamma_{*t}(e_{1, t})
\end{equation}
Writing this in words, the tangent vector to \(\varphi \circ \gamma\) at \(t\) is equal to the linear mapping between tangent spaces \(\varphi_{*\gamma(t)}: \reals^n_{\gamma(t)} \to \reals^p_{\varphi(\gamma(t))}\) applied to the tangent vector to \(\gamma\) at \(t\).

We're now going to consider functions which, for every point in \(\reals^n\), give us a vector in the tangent space at that point. And that's what's called a \textit{vector field}. Formally,
\begin{definition}
  A \underline{vector field} \(F\) on \(\reals^n\) is a function such that for every point \(a\), \(F(a) \in \reals^n_a\), i.e.
  \begin{equation}
    F(a) = (F_1(a),...,F_n(a))_a
  \end{equation}
\end{definition}
This trivially gives the following definition
\begin{definition}
  A vector field \(F\) is \(\mc{C}^r\) if each component \(F_i\) is \(\mc{C}^r\).
\end{definition}
Now, why is it that we really want to think of these things as having values in different tangent spaces? Let's look at some examples:
\begin{enumerate}

  \item In the plane, let's look at \(F(x, y) = (x, y)_{(x, y)}\), i.e. the vector \((x, y)\) ``pointing out of'' \(x, y\). In terms of the basis,
  \begin{equation}
    xe_{1, (x, y)} + ye_{2, (x, y)}
  \end{equation}
  We can imagine plotting this with every point having, coming out of it, the line between it and the origin, with a ``source'' at the origin. Imagining these to be the velocity vectors of a particle at each point, the points accelerate outwards from the origin with increasing rapidity.

  \item Consider now \(F(x, y) = (x, -y)\). If we trace a particle following the velocity vectors, this gives us a ``saddle'' shape.

  \item Consider the following important example
  \begin{equation}
    F(x) = \grad f(x) = \left(\prt{f}{x_1}(x),...,\prt{f}{x_n}\right)_x
  \end{equation}
  \(F\) is \(\mc{C}^r\) if \(f\) is \(\mc{C}^{r + 1}\). Finding if such an \(f\) exists is an interesting problem in differential equations. We can think of it as finding a potential function whose gradient at each point is equal to the vector field.

\end{enumerate}
We can apply operations on vectors pointwise to vector fields:
\begin{equation}
  (F + G)(x) = F(x) + G(x)
\end{equation}
\begin{equation}
  \ip{F}{G}(x) = \ip{F(x)}{G(x)}
\end{equation}
\begin{equation}
  (f \cdot F)(x) = f(x)F(x)
\end{equation}
In \(\reals^n\) we can take the cross product
\begin{equation}
  (F_1 \times ... \times F_{n - 1})(x) = F_1(x) \times ... \times F_{n - 1}(x)
\end{equation}
There are several very important classical operations we're going to be interested in, chief of them being \(\grad\), \(\Div\) and \(\curl\).
\begin{definition}
 We define the \underline{divergence of \(F\)}, \(\Div F\), as
 \begin{equation}
  \Div F = \sum_{i = 1}^n\D_iF_i = \sum_{i = 1}^n\prt{F_i}{x_i}
 \end{equation}
 This is often considered to be the ``inner product'' of \(F\) with a certain \underline{operator},
 \begin{equation}
  \ip{\nabla}{F}
 \end{equation}
 where
 \begin{equation}
  \nabla = \sum\prt{}{x_i}e_i
 \end{equation}
\end{definition}
\begin{definition}
  We define the \underline{curl of \(F\)} (in \(\reals^3\)), \(\curl F\), as
  \begin{equation}
    \curl F = \nabla \times F = (\D_2F_3 - \D_3F_2)e_1 + (\D_3F_1 - \D_1F_2)e_2 + (\D_1F_2 - \D_2F_1)e_3
  \end{equation}
\end{definition}
We're going to prove a very general version of Stokes' theorem, but we want to see what holds in these special cases. Maybe that's enough for today

\end{document}
