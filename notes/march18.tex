\documentclass{article}
\usepackage[utf8]{inputenc}

\title{MAT257 Notes}
\author{Jad Elkhaleq Ghalayini}
\date{March 18 2019}

\usepackage{amsmath}
\usepackage{amssymb}
\usepackage{amsthm}
\usepackage{mathtools}
\usepackage{enumitem}
\usepackage{graphicx}
\usepackage{cancel}

\usepackage[margin=1in]{geometry}

\newtheorem{theorem}{Theorem}
\newtheorem{lemma}{Lemma}
\newtheorem{definition}{Definition}
\newtheorem*{corollary}{Corollary}
\newtheorem{exercise}{Exercise}
\newtheorem{claim}{Claim}
\newtheorem{proposition}{Proposition}

\DeclareMathOperator{\Int}{Int}
\DeclareMathOperator{\grad}{grad}
\DeclareMathOperator{\Div}{div}
\DeclareMathOperator{\curl}{curl}
\DeclareMathOperator{\Ker}{Ker}
\DeclareMathOperator{\Ima}{Im}
\DeclareMathOperator{\Vol}{vol}
\DeclareMathOperator{\D}{D}

\newcommand{\reals}[0]{\mathbb{R}}
\newcommand{\nats}[0]{\mathbb{N}}
\newcommand{\ints}[0]{\mathbb{Z}}
\newcommand{\rationals}[0]{\mathbb{Q}}
\newcommand{\brac}[1]{\left(#1\right)}
\newcommand{\sbrac}[1]{\left[#1\right]}
\newcommand{\mc}[1]{\mathcal{#1}}
\newcommand{\eval}[3]{\left.#3\right|_{#1}^{#2}}
\newcommand{\ip}[2]{\left\langle#1,#2\right\rangle}
\newcommand{\prt}[2]{\frac{\partial #1}{\partial #2}}
\newcommand{\mb}[1]{\mathbf{#1}}
\newcommand{\hlfspc}[0]{\mathbb{H}}
\newcommand{\loint}[0]{\operatorname{L}\int}
\newcommand{\hiint}[0]{\operatorname{U}\int}
\newcommand{\indic}[1]{\chi_{#1}}

\begin{document}

\maketitle

\section{Stokes' Theorem}

We now want to try to prove the general form of Stokes' theorem:
\begin{theorem}[Stokes' Theorem]

  Let \(M\) be a compact oriented \(k\)-dimensional manifold with boundary (which is at least
  \(\mc{C}^2\)) and \(\omega\) be a \((k - 1)\)-form on \(M\) (which is at least \(\mc{C}^1\)). Then
  \begin{equation}
    \int_Md\omega = \int_{\partial M}\omega
    \label{stokes}
  \end{equation}
  where \(\partial M\) has induced orientation
\end{theorem}
The problem is, the integrals in equation \ref{stokes} are yet to be defined. So that's what we're going to be working up to today.
Consider a singular \(p\)-cube \(c: [0, 1]^k \to M\) in \(M\). If \(\omega\) is a \(p\)-form on \(M\), then we define
\begin{equation}
  \int_c\omega = \int_{[0, 1]^p}c^*\omega
\end{equation}
Integrals over \(p\)-chains are defined as before. In the case that \(p = k\) (e.g. in theorem) we'll assume that our \(k\)-cubes \(c:[0, 1]^k \to M\) satisfy the following condition: there is a coordinate chart \(\xi: W \to M\) such that \([0, 1]^k \subset W\) and
\begin{equation}
  c = \xi|_{[0, 1]^k}
\end{equation}
As a mapping, \(c\) is orientation preserving if and only if \(\xi\) is orientation preserving.
\begin{lemma}
  Let \(M\) be an oriented \(k\)-dimensional manifold (with or without boundary), and let \(c_1, c_2: [0, 1]^k \to M\) be orientation-preserving singular \(k\)-cubes, with the above assumption holding. If \(\omega\) is a \(k\)-form on \(M\) such that \(\omega = 0\) outside \(c_1([0, 1]^k) \cap c_2([0, 1]^k)\) then
  \begin{equation}
    \int_{c_1}\omega = \int_{c_2}\omega
  \end{equation}
  \label{cind}
\end{lemma}
\begin{proof}
  By definition, we have that
  \begin{equation}
    \int_{c_1}\omega = \int_{[0, 1]^k}c_1^*\omega
    \label{defncube}
  \end{equation}
  Using the fact that
  \begin{equation}
    c_1 = c_2 \circ (c_2^{-1} \circ c_1)
    \label{composedinv}
  \end{equation}
  Of course, ordinarily we can't talk about inverses when we're talking about cubes, but \(c_1, c_2\) have inverses since they are assumed to be the restrictions of a diffeomorphism. If you want, you can even write that
  \begin{equation}
    c_1^*\omega = \xi^*\omega
  \end{equation}
  Plugging equation \ref{composedinv} into equation \ref{defncube}, we obtain
  \begin{equation}
    \int_{[0, 1]^k}(c_2^{-1} \circ c_1)^*(c_2^*\omega)
  \end{equation}
  We have that
  \begin{equation}
    c_2^*\omega = f(\omega)dx_1 \wedge ... \wedge dx_k
  \end{equation}
  Let \(g = c_2^{-1} \circ c_1\). We then have that
  \begin{equation}
    (c_1^{-1} \circ c_2)^*(c_2^*\omega) = g^*(f dx_1 \wedge ... \wedge dx_k) = (f \circ g)\det g'dx_1 \wedge ... \wedge dx_k
  \end{equation}
  Suince \(g\) is orientation preserving, we get that \(\deg g' = |\det g'|\), implying the above is equal to
  \begin{equation}
    (f \circ g)|\det g'|dx_1 \wedge ... \wedge dx_k = \int_{c_1}\omega
  \end{equation}
  as desired. TODO: check

\end{proof}
We now want to define the integral of a form over a manifold:
\begin{definition}
  Let \(M\) be an oriented \(k\)-dimensional manifold and \(\omega\) be a \(k\)-form on \(M\).
  \begin{enumerate}

    \item If there is an orientation preserving \(k\)-cube \(c\) on \(M\) such that \(\omega = 0\) outside \(c([0, 1]^j)\) we define
    \begin{equation}
      \int_M\omega \int_c\omega
    \end{equation}
    which is independent of the choice of \(c\) by Lemma \ref{cind} as long as \(\omega\) vanishes outside it.

    \item In the general case, there exists an open cover \(\mc{O}\) of \(M \subset \reals^n\) such that for all \(U \in \mc{O}\), thee is an orientation preserving \(k\)-cube \(c_U\) in \(M\) such that
    \begin{equation}
      U \cap M \subset c_U([0, 1]^k)
    \end{equation}
    Now, let's let \(\Phi\) be a \(\mc{C}^2\) partition of unity subordinate to \(\mc{O}\). We define
    \begin{equation}
      \int_M\omega = \sum_{\varphi \in \Phi}\int_{M}\varphi \circ \omega
    \end{equation}
    If \(M\) is compact, this sum is finite, but in general, the definition still works out.

  \end{enumerate}
\end{definition}

\end{document}
