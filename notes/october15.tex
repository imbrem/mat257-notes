\documentclass{article}
\usepackage[utf8]{inputenc}

\title{MAT257 Notes}
\author{Jad Elkhaleq Ghalayini}
\date{October 15 2018}

\usepackage{amsmath}
\usepackage{amssymb}
\usepackage{amsthm}
\usepackage{mathtools}
\usepackage{enumitem}
\usepackage{graphicx}
\usepackage{cancel}

\usepackage[margin=1in]{geometry}

\newtheorem{theorem}{Theorem}
\newtheorem{lemma}{Lemma}
\newtheorem{definition}{Definition}
\newtheorem*{corollary}{Corollary}
\newtheorem{exercise}{Exercise}

\newcommand{\reals}[0]{\mathbb{R}}
\newcommand{\nats}[0]{\mathbb{N}}
\newcommand{\ints}[0]{\mathbb{Z}}
\newcommand{\rationals}[0]{\mathbb{Q}}
\newcommand{\brac}[1]{\left(#1\right)}
\newcommand{\sbrac}[1]{\left[#1\right]}
\newcommand{\mc}[1]{\mathcal{#1}}
\newcommand{\eval}[3]{\left.#3\right|_{#1}^{#2}}
\newcommand{\ip}[2]{\left\langle#1,#2\right\rangle}
\newcommand{\prt}[2]{\frac{\partial #1}{\partial #2}}

\begin{document}

\maketitle

\section*{Implicit Function Theorem}

Assume we have a function \(F(x, y) = 0\). Can we solve for \(y\) as a function \(y = g(x)\) near a point \((a, b)\) such that \(F(a, b) = 0\)?
For example, assume \(f(x, y) = x^2 + y^2 - 1\).
This relationship, of course, defines the unit circle. Can we solve for \(y\) at a function of \(x\) at any point on the unit circle? No: we can't do so where \(y = 0\), we can only get \(x\) as a function of \(y\).
If we pick a point \((a, b) \in \reals^2\) however, and assume \(a \neq \pm 1\), then there are open intervals \(I, J\) such that for every \(x \in I\), there is a unique \(y = g(x) \in J\) such that \(f(x, y) = 0\).

Let's restrict our attention to a small enough interval such that \(b > 0\). Of course, then, we can explicitly write down what the solution is:
\[g(x) = \sqrt{1 - x^2}\]

Sometimes in this context people say that one of the variables is dependent and the other is independent. Which is which? How should you understand that language? Well, the dependent variable is the one that is determined by the independent one you specify. But that notion, it depends on where you are: at the poles \(x = \pm 1\), we can only solve for \(x\), whereas at the poles \(y \pm 1\), we can only solve for \(y\).

Furthermore, we may not be able to find a nice formula for \(g\). However, by what's known as ``implicit differentiation,'' we can find a formula for the \textit{derivative of} \(g\).

\section*{Implicit Differentiation}
Let's assume we have a function
\[F(x, g(x)) = 0\]
Whether or not we have a formula for \(g\), we can use the chain rule to obtain
\[\frac{d}{dx}F(x, g(x)) = \partial_1(x, g(x)) + \partial_2(x, g(x))g'(x) = 0\]
So we find that
\[g'(x) = -\frac{\partial_1(x, g(x))}{\partial_2(x, g(x))}\]
Now this formula depends on \(g\). Is this weird or expected? Well, it makes sense. Consider the example above: we might have two solutions (points on the circle) on top of each other for a given \(x\)-coordinate. This formula works for \textit{both} of them.

\end{document}
