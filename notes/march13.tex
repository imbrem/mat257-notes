\documentclass{article}
\usepackage[utf8]{inputenc}

\title{MAT257 Notes}
\author{Jad Elkhaleq Ghalayini}
\date{March 13 2019}

\usepackage{amsmath}
\usepackage{amssymb}
\usepackage{amsthm}
\usepackage{mathtools}
\usepackage{enumitem}
\usepackage{graphicx}
\usepackage{cancel}

\usepackage[margin=1in]{geometry}

\newtheorem{theorem}{Theorem}
\newtheorem{lemma}{Lemma}
\newtheorem{definition}{Definition}
\newtheorem*{corollary}{Corollary}
\newtheorem{exercise}{Exercise}
\newtheorem{claim}{Claim}
\newtheorem{proposition}{Proposition}

\DeclareMathOperator{\Int}{Int}
\DeclareMathOperator{\grad}{grad}
\DeclareMathOperator{\Div}{div}
\DeclareMathOperator{\curl}{curl}
\DeclareMathOperator{\Ker}{Ker}
\DeclareMathOperator{\Ima}{Im}
\DeclareMathOperator{\Vol}{vol}
\DeclareMathOperator{\D}{D}

\newcommand{\reals}[0]{\mathbb{R}}
\newcommand{\nats}[0]{\mathbb{N}}
\newcommand{\ints}[0]{\mathbb{Z}}
\newcommand{\rationals}[0]{\mathbb{Q}}
\newcommand{\brac}[1]{\left(#1\right)}
\newcommand{\sbrac}[1]{\left[#1\right]}
\newcommand{\mc}[1]{\mathcal{#1}}
\newcommand{\eval}[3]{\left.#3\right|_{#1}^{#2}}
\newcommand{\ip}[2]{\left\langle#1,#2\right\rangle}
\newcommand{\prt}[2]{\frac{\partial #1}{\partial #2}}
\newcommand{\mb}[1]{\mathbf{#1}}
\newcommand{\hlfspc}[0]{\mathbb{H}}
\newcommand{\loint}[0]{\operatorname{L}\int}
\newcommand{\hiint}[0]{\operatorname{U}\int}
\newcommand{\indic}[1]{\chi_{#1}}

\begin{document}

\maketitle

Let's take off from last time: assume \(\omega\) is a \(p\)-form on a \(\mc{C}^{r+1}\) submanifold \(M\) of \(\reals^n\). That means that \(\omega\) maps each point \(a \in M\) to an alternating \(p\)-tensor \(\omega(a)\) on the tangent space to the manifold at \(a\), i.e.
\begin{equation}\omega : a \mapsto \omega(a) \in \Omega^p(M_a)\end{equation}
Consider two coordinate charts
\begin{equation}\varphi: W \to M, \psi: V \to M\end{equation}
We have, on the intersection \(\varphi(W) \cap \psi(V)\),
\begin{equation}\psi = \varphi \circ (\varphi^{-1} \circ \psi) \implies \psi^*\omega = (\varphi^{-1} \circ \psi)^*(\varphi^*\omega)\end{equation}
We used this to conclude that \(\varphi^*\) is \(\mc{C}^r\) if and only if \(\psi^*\) is \(\mc{C}^r\). So we say \(\omega\) is \(\mc{C}^r\) if \(\varphi^*\omega\) is \(\mc{C}^r\) for every coordinate chart \(\varphi\).
\begin{proposition}
  If \(\omega\) is a \(\mc{C}^r\) \(p\)-form on \(M\) then there is a unique \(\mc{C}^{r - 1}\) \((p + 1)\) form \(d\omega\) on \(M\) such that
  \begin{equation}
    \varphi^*(d\omega) = d(\varphi*\omega)
  \end{equation}
  for every coordinate chart \(\varphi\)
\end{proposition}
\begin{proof}
  Given \(\varphi\), we define \(d\omega\) on \(\varphi(W)\) by the formula
  \begin{equation}
    d\omega(a)(v_1,...,v_{p + 1}) = d(\varphi^*\omega)(x)(w_1,...,w_{p + 1})
    \label{dphi}
  \end{equation}
  where \(a = \varphi(x)\) and each \(w_i\) is the thing that the tangent mapping takes to \(v_i\), i.e.
  \begin{equation}
    v_i = \varphi_{*x}(w_i)
  \end{equation}
  For the skeptics among you, let's call equation \ref{dphi} \(d^\varphi\omega\), since it might depend on \(\varphi\). Last time, we observed that \(d^\varphi\omega\) is the unique \(p + 1\) form such that
  \begin{equation}
    \varphi^*(d^\varphi\omega) = d(\varphi^*\omega) \label{pullback}
  \end{equation}
  In the overlap of two charts \(\varphi: W \to \reals^n\), \(\psi: V \to \reals^n\), we have
  \begin{equation}
    \psi^*(d^\varphi\omega) = ((\varphi^{-1} \circ \psi)^* \circ \varphi^*)(d^\varphi\omega) = (\varphi^{-1} \circ \psi^{-1})^*d(\varphi^*\omega) = d((\varphi^{-1} \circ \psi)^* \circ \varphi^*\omega) = d(\psi^*\omega)
  \end{equation}
  But remember, \(d^\psi\omega\) was the unique thing satisfying equation \ref{pullback} substituting \(\psi\) for \(\varphi\). And hence we have that
  \begin{equation}
    d^\varphi\omega = d^\psi\omega
  \end{equation}
  So we can just call this \(d\omega\).
\end{proof}

\section{Orientiation of a sumbanifold \(M\) of \(\reals^n\)}
means there is a \underline{consistent} orientiation \(\mu_x\) on every tangent space \(M_x \subset \reals^k\). This means for every coordinate system \(\varphi: W \to \reals^n\), any two points \(a, b \in W\),
\begin{equation}
  [\varphi_{*a}(e_{1,a}),...,\varphi_{*a}(e_{k,a})] = \mu_{\varphi(a)} \iff [\varphi_{*b}(e_{1,b}),...,\varphi_{*b}(e_{k,b})] = \mu_{\varphi(b)}]
  \label{forevery}
\end{equation}
\begin{definition}
  An \underline{orientable manifold} is a manifold \(M\) for which it's possible to choose consistent orientations \(\mu_x\) at every point.
\end{definition}
Suppose the orientations \(\mu_x\) can be chosen consistently. Given any coordinate chart, we're going to call it either \underline{orientation preserving} or \underline{orientation reversing}. Specifically,
\begin{definition}
  A coordinate chart \(\varphi: W \to \reals^k\) is \underline{orientation preserving} if
  \begin{equation}
    [\varphi_{*a}(e_{1,a}),...,\varphi_{*a}(e_{k, a})] = \mu_{\varphi(a)}
  \end{equation}
  for one point \(a\) or every point \(a\) (these are equivalent by equation \ref{forevery}).
  If \(\varphi: W \to \reals^k\) is \underline{orientation reversing} (i.e. not orientation preserving), then if \(T: \reals^k \to \reals^k\) is a linear transformation with \(\det T < 0\), then \(\varphi \circ T\) is orientation preserving.
\end{definition}
So, every orientable manifold can be covered by orientation preserving coordinate charts (i.e. orientation preserving with respect to a consistent choice of orientations). Suppose now \(\varphi, \psi\) are two coordinate systems such that \(\varphi(a) = \psi(b)\) and
\begin{equation}
  [\varphi_{*a}(e_{1,a}),...,\varphi_{*b}(e_{k, a})] = \mu_x = [\psi_{*b}(e_{1,b}),...,\psi_{*b}(e_{k,b})]
\end{equation}
then \(\det(\varphi^{-1}, \psi)'(b) > 0\) (at every point in the overlap \(\varphi(W) \cap \psi(V)\)). We now obtain the following definitions:
\begin{definition}
If \(M\) is orientable, then a particular choice of consistent orientations \(\mu_x\) for \(x \in M\) is called an \underline{orientation of \(M\)}. An \underline{oriented manifold} \(M\) is a manifold together with an orientation.
\end{definition}
Let's now look at an example: the Möbius strip is \underline{not} orientable:s
\begin{proof} (Sketch): consider an orientation at one point along the strip. Move it along the strip until it returns back to where it started. Now it's reversed, so we have an inconsistency.
\end{proof}
Obviously, this is not very rigorous, but we'll look at another approach next time.

\end{document}
