\documentclass{article}
\usepackage[utf8]{inputenc}

\title{MAT257 Notes}
\author{Jad Elkhaleq Ghalayini}
\date{November 23 2018}

\usepackage{amsmath}
\usepackage{amssymb}
\usepackage{amsthm}
\usepackage{mathtools}
\usepackage{enumitem}
\usepackage{graphicx}
\usepackage{cancel}

\usepackage[margin=1in]{geometry}

\newtheorem{theorem}{Theorem}
\newtheorem{lemma}{Lemma}
\newtheorem{definition}{Definition}
\newtheorem*{corollary}{Corollary}
\newtheorem{exercise}{Exercise}
\newtheorem{claim}{Claim}

\DeclareMathOperator{\Int}{Int}
\DeclareMathOperator{\grad}{grad}
\DeclareMathOperator{\Ker}{Ker}
\DeclareMathOperator{\Ima}{Im}

\newcommand{\reals}[0]{\mathbb{R}}
\newcommand{\nats}[0]{\mathbb{N}}
\newcommand{\ints}[0]{\mathbb{Z}}
\newcommand{\rationals}[0]{\mathbb{Q}}
\newcommand{\brac}[1]{\left(#1\right)}
\newcommand{\sbrac}[1]{\left[#1\right]}
\newcommand{\mc}[1]{\mathcal{#1}}
\newcommand{\eval}[3]{\left.#3\right|_{#1}^{#2}}
\newcommand{\ip}[2]{\left\langle#1,#2\right\rangle}
\newcommand{\prt}[2]{\frac{\partial #1}{\partial #2}}
\newcommand{\mb}[1]{\mathbf{#1}}
\newcommand{\hlfspc}[0]{\mathbb{H}}

\begin{document}

\maketitle

\section{Manifolds with boundary}


Recall the following definitions:
\begin{definition}
  A half space \(\hlfspc^k \subset \reals^k\) is given by
  \[\{x = (x_1,...,x_k) : x_k \geq 0\}\]
\end{definition}
\begin{definition}
  A subset \(M \subseteq \reals^n\) is a \(k\)-dimensional \(\mc{C}^r\) \underline{manifold with boundary} if for every \(a \in M\), there is an open neighborhood \(U\) of \(a\) in \(\reals^n\), an open subset \(V\) of \(\reals^n\) and a \(\mc{C}^r\) diffeomorphism \(h: U \to V\) such that either
  \begin{enumerate}[label=(\arabic*)]

    \item The usual condition for a manifold,
    \[h(M \cap U) = V \cap (\reals^k \times \{\mb{0} \in \reals^{n - k}\})\]
    \label{mnfcnd}

    \item The manifold looks like a half space around \(a\), i.e.
    \[h(M \cap U) = V \cap (\hlfspc^k \times \{\mb{0} \in \reals^{n - k}\}) = \{y = (y_1,...,y_n) : y_1,...,y_k \geq 0 \land y_{k + 1} = ... = y_n = 0\}\] \label{bndcnd}

  \end{enumerate}
  \label{mwb}
\end{definition}
One thing you could ask is whether both conditions \ref{mnfcnd}, \ref{bndcnd} could be satisfied by the same point \(a\). The answer is no. To show why, suppose \(h_1: U_1 \to V_1\) satisfies \ref{mnfcnd} and \(h_2: U_2 \to V_2\) satisfies \ref{bndcnd} for some \(a\). How can we get a contradiction? If we take
\[h_2 \circ h_1^{-1}: V_1 \to V_2\]
we get a \(\mc{C}^r\) diffeomorphism in an open set containing \(a\) taking an open subset of \(\reals^k\) onto an open subset of \(\hlfspc^k\) which is \textit{not} open in \(\reals^k\). What we're contradicting here is the inverse function theorem, which says there would be open neighborhoods \(h_1(a)\) and \(h_2(a)\) which are mapped together with an inverse. Alternatively, we could use the topological arguments that diffeomorphisms, which are homeomorphisms, are open maps.

Once we've established this result, we can really distinguish these two kinds of points. That means, precisely, that we can make the following definitions
\begin{definition}
  The \underline{boundary} of \(M\), written \(\partial M\), is the set of points satisfying condition \ref{bndcnd} in definition \ref{mwb}
\end{definition}

We're going to be interested in manifolds because they generalize the idea of intervals with endpoints in the real line. In fact, an interval with endpoints is a one dimensional manifold with boundary.

\end{document}
