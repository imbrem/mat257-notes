\documentclass{article}
\usepackage[utf8]{inputenc}

\title{MAT257 Notes}
\author{Jad Elkhaleq Ghalayini}
\date{February 15 2019}

\usepackage{amsmath}
\usepackage{amssymb}
\usepackage{amsthm}
\usepackage{mathtools}
\usepackage{enumitem}
\usepackage{graphicx}
\usepackage{cancel}

\usepackage[margin=1in]{geometry}

\newtheorem{theorem}{Theorem}
\newtheorem{lemma}{Lemma}
\newtheorem{definition}{Definition}
\newtheorem*{corollary}{Corollary}
\newtheorem{exercise}{Exercise}
\newtheorem{claim}{Claim}
\newtheorem{proposition}{Proposition}

\DeclareMathOperator{\Int}{Int}
\DeclareMathOperator{\grad}{grad}
\DeclareMathOperator{\Div}{div}
\DeclareMathOperator{\curl}{curl}
\DeclareMathOperator{\Ker}{Ker}
\DeclareMathOperator{\Ima}{Im}
\DeclareMathOperator{\Vol}{vol}
\DeclareMathOperator{\D}{D}

\newcommand{\reals}[0]{\mathbb{R}}
\newcommand{\nats}[0]{\mathbb{N}}
\newcommand{\ints}[0]{\mathbb{Z}}
\newcommand{\rationals}[0]{\mathbb{Q}}
\newcommand{\brac}[1]{\left(#1\right)}
\newcommand{\sbrac}[1]{\left[#1\right]}
\newcommand{\mc}[1]{\mathcal{#1}}
\newcommand{\eval}[3]{\left.#3\right|_{#1}^{#2}}
\newcommand{\ip}[2]{\left\langle#1,#2\right\rangle}
\newcommand{\prt}[2]{\frac{\partial #1}{\partial #2}}
\newcommand{\mb}[1]{\mathbf{#1}}
\newcommand{\hlfspc}[0]{\mathbb{H}}
\newcommand{\loint}[0]{\operatorname{L}\int}
\newcommand{\hiint}[0]{\operatorname{U}\int}
\newcommand{\indic}[1]{\chi_{#1}}

\begin{document}

\maketitle

Let's begin with a \(\mc{C}^r\) mapping \(f: \reals^n \to \reals^p\). Well first of all, we know what the tangent mapping is, or rather we know what the derivative is. And we want to interpret this derivative as a linear mapping induced by \(f\)
\begin{equation}
  f_{*a}: \reals^n_a \to \reals^p_{f(a)}, f_{*a}(v_a) = Df(a)(v)_{f(a)}
\end{equation}
Now, by duality, we want to say that \(f_{\*a}\) induces in the other direction a mapping that we'll call \(f^*_a\) that takes alternating \(k\)-tensors on the tangent space \(\reals^p_{f(a)}\) to alternating \(k\)-tensors on the tangent space \(\reals^n_a\), i.e.
\begin{equation}
  f_a^*: \Omega^k(\reals^p_{f(a)}) \to \Omega^k(\reals^n_a)
\end{equation}
Let's try the case where \(k = 1\). In this case, alternating 1-tensors are just 1-tensors, which are just elements of the dual space, i.e.
\begin{equation}
  \Omega^1(\reals^n_a) = \mc{T}^1(\reals^n_a) = (\reals^n_a)^*
\end{equation}
So, if \(T \in (\reals^p_{f(a)})^*\), we can write
\begin{equation}
  f_a^*(T)(v_a) = T(f_{*a}(v_a))
\end{equation}
In general, we do the exact same thing for \(\omega \in \Omega^k(\reals^p_{f(a)})\):
\begin{equation}
  f_a^*(\omega)(v_{1,a},...,v_{k,a}) = \omega(f_{*a}(v_1),...,f_{*a}(v_k))
\end{equation}
Now, we want to show that \(f\) induces a mapping \(f^*\) inducing a linear mapping whih takes, instead of pointwise objects as above, \(k\)-forms on \(\reals^p\) to \(k\)-forms on \(\reals^n\), with a \(k\)-form on \(\reals^p\) taking the form
\begin{equation}
  \omega: b \in \reals^p \mapsto \omega(b) \in \Omega^k(\reals^p_b)
\end{equation}
So we can write
\begin{equation}
  (f^*\omega)(a) = f^*_a(\omega(f(a)))
  \label{novs}
\end{equation}
which is the same as saying
\begin{equation}
  (f^*\omega)(a)(v_{1, a},...,v_{k, a}) = \omega(f(a))(f_{*a}(v_{1,a}),...,f_{*a}(v_{n,a}))
\end{equation}
that is, just repeating what the definition is about. So this is the formal definition. I understand that everyone would like to forget this as soon as possible, so I'm going to try to help.
\begin{proposition}
  Let \(f: \reals^n \to \reals^p\) be \(\mc{C}^r\). Then
  \begin{equation}
    f^*(dy_i) = \sum_{j = 1}^n\prt{f_i}{x_j}dx_j = df_i = d(y_i \circ f)
  \end{equation}
  where
  \begin{equation}
    dy_i = \sum\prt{y_i}{x_j}dx_j
  \end{equation}
  Recall also that
  \begin{equation}
    f_*\left(\prt{}{x_j}\right) = \sum\prt{f_j}{x_i}\prt{}{y_j}
  \end{equation}
\end{proposition}
Before we prove this, recall that equation \ref{novs} is true for every \(k\), including \(k = 0\). And what's a 0-form? It's just a function. So in this case, with a function (zero form) \(g\),
\begin{equation}
  f^*(g) = g \circ f
\end{equation}
Let's get to the proof:
\begin{proof}
  By definition,
  \begin{equation}
    f^*(dy_i)(a)(v_a) = dy_i(f(a))(f_{*a}v_a)
    \label{mprod}
  \end{equation}
  Now, what is the right hand side here? It's the matrix of partial derivatives of \(f(a)\) applied to \(v_a\). So if \(v_a = (v_1,...,v_n)\), then equation \ref{mprod} is equal to
  \begin{equation}
    dy_i(f(a))\left(\sum_{k = 1}^p\mb{e}_{k, f(a)}
      \sum_{j = 1}^n\prt{f_k}{x_j}(a)v_j\right)
      = \sum_{j = 1}^n\prt{f_i}{x_j}(a)v_j
      = \sum_{j = 1}^n\prt{f_i}{x_j}(a)dx_j(a)v_a
  \end{equation}
  Since this is true for any \(v_a\), it follows that
  \begin{equation}
    f^*(dy_i) = \sum_{j = 1}^n\prt{f_i}{x_j}dx_j
  \end{equation}
  as desired.
\end{proof}
Let's do an example: assume \(gx\) is a \(\mc{C}^r\) function on \(\reals^n\):
\begin{equation}
  f^*(dg) = f^*\left(\sum_{i = 1}\prt{g}{y_i}dy_i\right) = \sum_{i = 1}^p\prt{g}{y_i} \circ ff^*(dy_i) = \sum_{i = 1}^n\prt{g}{y_i} \circ f\sum_{j = 1}^n\prt{f}{x_j}dx_j
\end{equation}
Changing the order of the summation and using the chain rule, we obtain
\begin{equation}
  \sum_{j = 1}^n\left(\sum_{i = 1}^n\prt{g}{y_i}\circ f \circ \prt{f_i}{x_j}\right)dx_j = \sum_{j = 1}^n\prt{(g \circ f)}{x_j}dx_j = d(g \circ f)
\end{equation}
Let's extend the proposition with some ``trivial'' facts:
\begin{proposition}
  Let \(f: \reals^n \to \reals^p\) be \(\mc{C}^r\). Then
  \begin{enumerate}

    \item \(f^*(\omega_1 + \omega_2) = f^*(\omega_1) + f^*(\omega_2)\), i.e. \(f^*\) is linear

    \item \(f^*(g \cdot w) = (g \circ f)f^*\omega\)

    \item \(f^*(\omega \wedge \eta) = f^*(\omega) \wedge f^*(\eta)\)

  \end{enumerate}
\end{proposition}
\begin{proof}
  Immediate from the definitions
\end{proof}
\begin{proposition}
  Suppose \(f: \reals^n \to \reals^p\). Then
  \begin{equation}
    f^*(gdx_1 \wedge ... \wedge dx_n) = g \circ f\det g' \cdot dx_1 \wedge ... \wedge dx_n
  \end{equation}
\end{proposition}
This is why the natural object of integration is not functions but rather differential forms. Because if we understand that the object we're integrating is a differential form, not a function, then the formula for change of variable is built into the definition of a differential form.
\begin{equation}
  \int gdx_1 \wedge ... \wedge dx_n = \int gdx_1...dx_n
\end{equation}
Let's get to the proof:
\begin{proof}
  Consider
  \begin{equation}
    f^*(dx_1 \wedge ... \wedge dx_n)(a)(e_{1,a},...,e_{n,a})
  \end{equation}
\end{proof}

\end{document}
