\documentclass{article}
\usepackage[utf8]{inputenc}

\title{MAT257 Notes}
\author{Jad Elkhaleq Ghalayini}
\date{September 16 2018}

\usepackage{amsmath}
\usepackage{amssymb}
\usepackage{amsthm}
\usepackage{mathtools}
\usepackage{enumitem}
\usepackage{graphicx}
\usepackage{cancel}

\usepackage[margin=1in]{geometry}

\newtheorem{theorem}{Theorem}
\newtheorem{lemma}{Lemma}
\newtheorem{definition}{Definition}
\newtheorem*{corollary}{Corollary}
\newtheorem{exercise}{Exercise}

\newcommand{\reals}[0]{\mathbb{R}}
\newcommand{\nats}[0]{\mathbb{N}}
\newcommand{\ints}[0]{\mathbb{Z}}
\newcommand{\rationals}[0]{\mathbb{Q}}
\newcommand{\brac}[1]{\left(#1\right)}
\newcommand{\sbrac}[1]{\left[#1\right]}
\newcommand{\mc}[1]{\mathcal{#1}}
\newcommand{\eval}[3]{\left.#3\right|_{#1}^{#2}}
\newcommand{\ip}[2]{\left\langle#1,#2\right\rangle}

\begin{document}

\maketitle

\section*{Compactness}

\begin{definition}
  A subset \(X\) of \(\reals^n\) is \underline{compact} if every open covering of \(X\) has a finite subcover.
\end{definition}

We're later going to prove a deep theorem about compactness:
\begin{theorem}
  A subset \(X\) of \(\reals^n\) is compact if and only if \(X\) is closed and bounded
\end{theorem}
We won't do this right now, because it'll involve some work, but we'll use something called the Heine-Borel Theorem: \([0, 1]\) is compact. We're going to prove this exactly the same way as we proved the ``Three Hard Theorems'' from first year calculus. It's a good exercise to try this.


Today, we'll do half of this theorem, the easy part: compact \textit{implies} closed and bounded.
\begin{lemma}
  If \(X \subset \reals^n\) is compact then \(X\) is closed and bounded.
\end{lemma}
\begin{proof}
  \begin{itemize}
    \item \(X\) is closed: to say that \(X\) is closed is of course the same thing as saying \(\reals^n \setminus X\) is open. To show this, we take any point in \(\reals^n \setminus X\), and show there's some ball centered at that point which is a subset of \(\reals^n \setminus X\).

    So let \(x \in \reals^n \setminus X\). We want to show that \(\exists \delta > 0\), \(B(a, \delta) \subset \reals^n \setminus X\) for some \(\delta > 0\). Let's just look at all possible balls centered at \(A\). Or we might as well just consider, for some \(k \in \nats\), the closed ball \(\overline{B(a, k^{-1})}\). Let's look at the complement of this closed ball,
    \[U_k = \reals^n \setminus \overline{B(a, k^{-1})}\]
    We have that
    \[\bigcup_kU_k = \reals^n\setminus\{a\}\]
    So \(\{U_k\}\) is an open cover of \(X\). But \(X\) is compact, so this open cover has a finite subcover. This fact tells us that \(\exists k\), \(X \subset U_k\). But \(U_k\) is the complement of the closed ball of radius \(\frac{1}{k}\), meaning the open ball
    \[B(a, k^{-1}) \subseteq \reals^n \setminus X\]

    We can clearly see that this implies \(\reals^n \setminus X\) is open, implying \(X\) is closed.

    \item \(X\) is bounded:  consider a cover of \(X\) by all open balls of radius 1 in \(\reals^n\).
  \end{itemize}
\end{proof}

\begin{theorem}
  If \(X \subset \reals^m\) is compact and \(f: X \to \reals^n\) is continuous then \(f(X)\) is pact
\end{theorem}
\begin{proof}
  Let \(\mc{O}\) be an open cover of \(f(X)\). For every \(U \in \mc{O}\), \(f^{-1}(U) = X \cap V_U\) where \(V_U\) is open in \(\reals^m\). So
  \[\{V_U : U \in \mc{O}\}\]
  is an open cover of \(X\). Since \(X\) is compact, there is a finite subcover
  \[V_{U_1},...,V_{U_k}\]
  So \(U_1,...,U_k\) cover \(f(X)\).
\end{proof}

\begin{theorem}[Extreme Value Theorem]
  A continuous function \(f: X \to \reals\) on a nonempty compact subset \(X\) of \(\reals^n\) attains a maximum and minimum
\end{theorem}
\begin{proof}
  Let \(M = \sup\{f(x) : x \in X\}\). \(M < \infty\) since \(f(X)\) is bounded. What if \(M \notin f(X)\)? Since \(f(X)\) is closed, we can draw a small interval around \(M\) which is not in \(f(X)\), contradicting the fact that \(M = \mbox{lub} f(X)\).

  The proof for minima is analogous.
\end{proof}

\begin{definition}
  The \(\epsilon\)-neighborhood of \(X\) is given by
  \[\bigcup_{x \in X}B(x, \epsilon) = \{y \in \reals^n : d(y, X) < \epsilon\}\]
\end{definition}

\begin{theorem}[\(\epsilon\)-neighborhood theorem]
  If \(X \subset U \subset \reals^n\) where \(X\) is compact and \(U\) is open, then there is \(\epsilon > 0\) such that the \(\epsilon\)-neighborhood of \(X\) lies in \(U\).
\end{theorem}
\begin{proof}
  Define \(f: X \to \reals\) by \(f(x) = d(x, \reals^n \setminus U)\). We showed this is a continuous function, which is strictly positive (since \(U\) is open). Hence, by the Extreme Value Theorem, it has a positive minimum. Take \(\epsilon\) to be this minimum.
\end{proof}

\end{document}
