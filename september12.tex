\documentclass{article}
\usepackage[utf8]{inputenc}

\title{MAT257 Notes}
\author{Jad Elkhaleq Ghalayini}
\date{September 12 2018}

\usepackage{amsmath}
\usepackage{amssymb}
\usepackage{amsthm}
\usepackage{mathtools}
\usepackage{enumitem}
\usepackage{graphicx}
\usepackage{cancel}

\usepackage[margin=1in]{geometry}

\newtheorem{theorem}{Theorem}
\newtheorem{lemma}{Lemma}
\newtheorem{definition}{Definition}

\newcommand{\reals}[0]{\mathbb{R}}
\newcommand{\nats}[0]{\mathbb{N}}
\newcommand{\ints}[0]{\mathbb{Z}}
\newcommand{\rationals}[0]{\mathbb{Q}}
\newcommand{\brac}[1]{\left(#1\right)}
\newcommand{\sbrac}[1]{\left[#1\right]}
\newcommand{\eval}[3]{\left.#3\right|_{#1}^{#2}}
\newcommand{\ip}[2]{\left\langle#1,#2\right\rangle}

\begin{document}

\maketitle

\section*{Functions on \(\reals^n\)}

You all know that \(\reals^n\) is the space of \(n\)-tuples of real numbers,
\[x = (x_1,...,x_n)\]
\(\reals^n\), of course, is a vector space with the operations
\[x + y = (x_1 + y_1,...,x_n + y_n)\]
\[ax = (ax_1,...,ax_n)\]
These are the vector space operations on \(\reals^n\) that everyone should be familiar with from linear algebra. We, however, are going to be even more interested in the \textit{metric} properties of such a metric space, and these are going to be dependent on the idea of norm that we talked about last time.

We defined a number of norms, including the norm
\[|x| = \sqrt{x_1^2 + ... + x_n^2}\]
Of course, this norm is intimately connected with the usual inner product on \(\reals^n\),
\[\ip{x}{y} = x_1y_1 + ... + x_ny_n\]
What are the basic relations between this and the vector space structure? These should be things you're familiar with, but let's recall.

Let's consider two vectors \(x, y \in \reals^n\). We know that
\begin{enumerate}
  \item \(|x| \geq 0\) and \(|x| = 0 \iff x = 0\)
  \item \(|\ip{x}{y}| \leq |x||y|\), with equality iff \(x, y\) are linearly dependent.
  \item \(|x + y| \leq |x| + |y|\), with equality iff one is zero or if they point in the same direction.
\end{enumerate}
\begin{proof}
\begin{itemize}

  \item [2.] We have equality if \(x, y\) are linearly dependent. But what if \(x, y\) are linearly independent. Then, by definition
  \[\forall \lambda \in \reals, y - \lambda x \neq 0\]
  \[\implies 0 < |y - \lambda x|^2
  = \ip{y}{y} - 2\lambda\ip{x}{y} + \lambda^2\ip{x}{x}\]
  \[= |y|^2 - 2\lambda\ip{x}{y} + \lambda^2|x|^2\]
  This is a quadratic polynomial in \(\lambda\) which is never 0, so hence has a discriminant
  \[4\ip{x}{y}^2 - 4|x|^2|y|^2 < 0 \implies \ip{x}{y}^2 < |x|^2|y|^2\]
  \[\implies |\ip{x}{y}| < |x||y|\]
  as desired.

  \item [3.] \(|x + y|^2  = \ip{x + y}{x + y} = |x|^2 + |y|^2 + 2\ip{x}{y}
  \leq |x|^2 + |y|^2 + 2|x||y| = (|x| + |y|)^2\).
\end{itemize}
\end{proof}

What if we have two vectors \(x\) and \(y\), and we want to define the angle between them. How should we do that? We can try

\begin{definition}
\[\angle(x, y) = \arccos\frac{\ip{x}{y}}{|x||y|}\]
\end{definition}
Why should this, however, be the definition? It comes from the elementary geometric relationship between the length of \(y - x\) and the lengths of \(x\), \(y\) when formed into a triangle:
\[|y - x|^2 = |x|^2 + |y|^2 - 2|x||y|\cos\theta\]
which can be rewritten to the familiar
\[\ip{x}{y} = |x||y|\cos\theta\]

\subsection*{Linear transformations from \(\reals^n \to \reals^m\)}

Usually, we represent linear transformations by a matrix. But the matrix, of course, corresponds to a choice of basis vectors. Let's recall what we usually denote the standard basis of \(\reals^n\)

\subsubsection*{Standard basis of \(\reals^n\)}

The standard basis of \(\reals^n\) is composed of \(e_1,...,e_n\), where each \(e_k\) is 0 everywhere except in the \(k^{th}\) place, where it is 1.

Let's assume \(T: \reals^n \to \reals^m\) is a linear transformation. What do we mean by the matrix of \(T\) with respect to the standard basis? Let
\[e_1,...,e_n \in \reals^n\]
\[f_1,...,f_m \in \reals^m\]
be the standard bases.

Our transformation is a matrix \(A\) that we multiply a certain vector \(x = (x_1,...,x_n) \in \reals^n\) by to get a vector \(y = (y_1,...,y_m) \in \reals^m\). Let's label the elements of \(A\) \(a_{11},...,a_{mn}\), getting
\[\begin{pmatrix} y_1 \\ \vdots \\ y_m \end{pmatrix} =
\begin{pmatrix}
  a_{11} & ... & a_{1n} \\
  \vdots & \ddots & \vdots \\
  a_{m1} & ... & a_{mn}
\end{pmatrix}\begin{pmatrix}
  x_1 \\ \vdots \\ x_n
\end{pmatrix}\]
But note we can write
\[x = \sum_{i = 1}^nx_ie_i\]
And note that
\[T(e_i) = \sum_{j = 1}^ma_{ji}f_j\]
i.e. the \(i^{th}\) column.

So what we're really doing is taking one sum of basis vectors into another.

Now what if we have a separate linear transformation? If \(S: \reals^m \to \reals^\ell\) is another linear transformation, that means we can compose them: we can apply \(S\) to \(T(x)\). Then
\[S \circ T: \reals^n \to \reals^\ell\]
Let's suppose \(S\) has a matrix \(B\). And what's the matrix of \(S \circ T\)? It is, of course, \(BA\).

Ok, finally, what about \textit{bounding} the size of a matrix applied to a vector? This is going to be a basic bound that we use all the time.

If \(T: \reals^n \to \reals^m\) is a linear transformation, then there is a constant \(M \geq 0\) such that
\[|Tx| \leq M|x|\]
Why?
\[T(x) = T\left(\sum x_ie_i\right) = \sum x_iT(e_i)\]
Let's estimate the norm:
\[|T(x)| \leq \sum_{i = 1}^n|x_iT(e_i)| = \sum_{i = 1}^n|x_i||T(e_i)|\]
So how can we use this to get that \(M\)? We can just take
\[C = \max_{i}|T(e_i)|\]
then
\[|T(x)| \leq C\sum|x_i|\]
This is not quite what we wanted, because this is not that single barred norm of \(x\). This was a different norm of \(x\). But what was the relationship between the two norms?
\[\sum|x_i| \leq \sqrt{n}|x|\]
So we have
\[|T(x) \leq \sqrt{n}C|x|\]

\(\reals^n\) with any of our norms is what's called a \textit{metric space}. All a metric space means is a set \(X\) together with a distance function
\[d: X \times X \to \reals, (x, y) \mapsto d(x, y)\]
such that
\begin{itemize}
  \item \(\forall x, y \in X, d(x, y) \geq 0\)
  \item \(\forall x, y, z \in X, d(x, z) \leq d(x, y) + d(y, z)\)
  \item \(d(x, y) = 0 \iff x = y\)
\end{itemize}

In this course, we're going to talk a little bit about the topology of \(\reals^n\), that is, talking about what are called open and closed sets, what the relationship between them is, and how this notion relates to continuity.

In fact, we're only going to be talking about a very special class of topological spaces: metric spaces. In fact, they're even a very special class of metric spaces: \(\reals^n\) or certain subsets of \(\reals^n\) with the Euclidean metric.

That's what we're going to be talking about tonight (note: I cannot attend, so no notes are available for this lecture): the topology of \(\reals^n\).

\end{document}
