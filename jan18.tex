\documentclass{article}
\usepackage[utf8]{inputenc}

\title{MAT257 Notes}
\author{Jad Elkhaleq Ghalayini}
\date{January 18 2019}

\usepackage{amsmath}
\usepackage{amssymb}
\usepackage{amsthm}
\usepackage{mathtools}
\usepackage{enumitem}
\usepackage{graphicx}
\usepackage{cancel}

\usepackage[margin=1in]{geometry}

\newtheorem{theorem}{Theorem}
\newtheorem{lemma}{Lemma}
\newtheorem{definition}{Definition}
\newtheorem*{corollary}{Corollary}
\newtheorem{exercise}{Exercise}
\newtheorem{claim}{Claim}

\DeclareMathOperator{\Int}{Int}
\DeclareMathOperator{\grad}{grad}
\DeclareMathOperator{\Ker}{Ker}
\DeclareMathOperator{\Ima}{Im}

\newcommand{\reals}[0]{\mathbb{R}}
\newcommand{\nats}[0]{\mathbb{N}}
\newcommand{\ints}[0]{\mathbb{Z}}
\newcommand{\rationals}[0]{\mathbb{Q}}
\newcommand{\brac}[1]{\left(#1\right)}
\newcommand{\sbrac}[1]{\left[#1\right]}
\newcommand{\mc}[1]{\mathcal{#1}}
\newcommand{\eval}[3]{\left.#3\right|_{#1}^{#2}}
\newcommand{\ip}[2]{\left\langle#1,#2\right\rangle}
\newcommand{\prt}[2]{\frac{\partial #1}{\partial #2}}
\newcommand{\mb}[1]{\mathbf{#1}}
\newcommand{\hlfspc}[0]{\mathbb{H}}
\newcommand{\loint}[0]{\operatorname{L}\int}
\newcommand{\hiint}[0]{\operatorname{U}\int}
\newcommand{\indic}[1]{\chi_{#1}}

\begin{document}

\maketitle

\section{Change of Variable Theorem}

\begin{theorem}
  Let \(A \subset \reals^n\) be open, \(g: A \to \reals^n\) be one to one, continuously differentiable and let, for all \(x \in A\), \(g'(x) \neq 0\). Then
  \begin{equation}
    f : g(A) \to \reals
  \end{equation}
  is integrable if and only if
  \begin{equation}
    f \circ g|\det g'|
  \end{equation}
  is integrable on \(A\). In this case,
  \begin{equation}
    \int_{g(A)}f = \int_Af \circ g|\det g'|
  \end{equation}
  \label{covtheorem}
\end{theorem}
Let's look at some examples, starting with polar coordinates: we use coordinates \(r \in \reals^+_0, \theta \in [0, 2\pi]\) and write
\begin{equation}
  x = r\cos\theta, y = r\sin\theta
\end{equation}
We have
\begin{equation}
  D = \prt{(x, y)}{(r, \theta)} = \begin{pmatrix}
    \cos\theta & -r\sin\theta \\
    \sin\theta & r\cos\theta
  \end{pmatrix} \implies \det D = r
\end{equation}
So we can write
\begin{equation}
  \iint_Af(x, y)dxdy = \iint_Af(r\cos\theta, r\sin\theta)rdrd\theta
\end{equation}
We now examine a corollary of Theorem \ref{covtheorem}.

\begin{corollary}
  Let \(A \subset C \subset \reals^n\) where \(A\) is open, \(C\) is compact and Jordan-measurable and \(C \setminus A\) has measure zero. If \(g\) is a continuously differentiable function from a neighborhood of \(C\) to \(\reals^n\) wich satisfies the conditions of theorem \ref{covtheorem} on \(A\), then
  \begin{equation}
    f: g(C) \to \reals
  \end{equation}
  is integrable if and only if
  \begin{equation}
    f \circ g|\det g'|
  \end{equation}
  is iintegrable on \(C\), and in this case
  \begin{equation}
    \int_{g(C)}f = \int_Cf \circ g|\det g'|
  \end{equation}
  \label{corcov}
\end{corollary}
\begin{lemma}
  Assume \(A \subset \reals^n\) is open and \(g: A \to \reals^n\) is continuously differentiable. If \(B \subset A\) has measure zero, then \(g(B)\) has measure zero.
\end{lemma}
\begin{proof}
  Enough to prove that \(g(B \cap C)\) has measure zero for any \(C \subset A\) compact, since \(A\) has an exhaustion by countably many compact sets \(C_1 \subset C_2 \subset ...\)

  To do so, remember that a countable intersection of measure 0 sets is measure 0. Using \(\mc{C}_1\), which is more than uniformly continuous, we have that
  \begin{equation}
    \forall x \in C, \forall y \in U, |g(x) - g(y)| \leq c|x - y|
  \end{equation}
  where \(U\) is some neighborhood of \(C\). So \(g\) maps a ball of radius \(\epsilon\) to a ball of radius \(c\epsilon\).
\end{proof}
We now proceed to prove Corollary \ref{corcov}
\begin{proof}
  \(g(C) \setminus g(A) \subseteq g(C \setminus A)\), and so it is of measure zero. We hence have that
  \begin{equation}
    \int_{g(A)}f = \int_{g(C)}f
  \end{equation}
  \begin{equation}
    \int_A(f \circ g)|\det g'| = \int_{C}(f \circ g)|\det g'|
  \end{equation}
  giving the desired equality by Theorem \ref{covtheorem}
\end{proof}
Let's move on to another example: what are called spherical coordinates. We use coordinates \(r \in \reals^+_0, \phi \in [0, 2\pi], \theta \in [0, \pi]\) where
\begin{equation}
  x = r\cos\phi\sin\theta, y = r\sin\phi\sin\theta, z = r\cos]theta
\end{equation}
We have... this is going to hurt...
\begin{equation}
  D = \prt{(x, y, z)}{(r, \theta, \phi)} = \begin{pmatrix}
    \cos\phi\sin\theta & r\cos\phi\cos\theta & -r\sin\phi\sin\theta \\
    \sin\phi\sin\theta & r\sin\phi\cos\theta & r\cos\phi\sin\theta \\
    \cos\theta & -r\sin\theta & 0
  \end{pmatrix} \implies \det D = r^2\sin\theta
\end{equation}

\end{document}
