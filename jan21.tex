\documentclass{article}
\usepackage[utf8]{inputenc}

\title{MAT257 Notes}
\author{Jad Elkhaleq Ghalayini}
\date{January 21 2019}

\usepackage{amsmath}
\usepackage{amssymb}
\usepackage{amsthm}
\usepackage{mathtools}
\usepackage{enumitem}
\usepackage{graphicx}
\usepackage{cancel}

\usepackage[margin=1in]{geometry}

\newtheorem{theorem}{Theorem}
\newtheorem{lemma}{Lemma}
\newtheorem{definition}{Definition}
\newtheorem*{corollary}{Corollary}
\newtheorem{exercise}{Exercise}
\newtheorem{claim}{Claim}

\DeclareMathOperator{\Int}{Int}
\DeclareMathOperator{\grad}{grad}
\DeclareMathOperator{\Ker}{Ker}
\DeclareMathOperator{\Ima}{Im}

\newcommand{\reals}[0]{\mathbb{R}}
\newcommand{\nats}[0]{\mathbb{N}}
\newcommand{\ints}[0]{\mathbb{Z}}
\newcommand{\rationals}[0]{\mathbb{Q}}
\newcommand{\brac}[1]{\left(#1\right)}
\newcommand{\sbrac}[1]{\left[#1\right]}
\newcommand{\mc}[1]{\mathcal{#1}}
\newcommand{\eval}[3]{\left.#3\right|_{#1}^{#2}}
\newcommand{\ip}[2]{\left\langle#1,#2\right\rangle}
\newcommand{\prt}[2]{\frac{\partial #1}{\partial #2}}
\newcommand{\mb}[1]{\mathbf{#1}}
\newcommand{\hlfspc}[0]{\mathbb{H}}
\newcommand{\loint}[0]{\operatorname{L}\int}
\newcommand{\hiint}[0]{\operatorname{U}\int}
\newcommand{\indic}[1]{\chi_{#1}}

\begin{document}

\maketitle

\section{Change of Variable}
Let's begin with some examples
\begin{enumerate}

  \item Find the volume of an ellipsoid
  \begin{equation}
    E = \left\{(x, y, z) \in \reals^3 : \frac{x^2}{a^2} + \frac{y^2}{b^2} + \frac{z^2}{c^2} \leq 1\right\}
  \end{equation}
  We can make a change of variable to ``transform'' this into a sphere. Specifically, we can use the transformation
  \begin{equation}
    g : \reals \to \reals^3, (x, y, z) \mapsto (x/a, y/b z/c)
  \end{equation}
  to take the ellipsoid to the unit ball, where \((u, v, w) = g(x, y, z)\). So
  \begin{equation}
    \iiint_{B_1(\mb{0})}1dudvdw = \iiint_{E}\frac{1}{abc}dxdydz = \frac{1}{abc}\iiint_{E}1dxdydz = \frac{4}{3}\pi \implies V(E) = \frac{4}{3}\pi abc
  \end{equation}
  Now what if we didn't know the volume of a ball of radius 1? We could figure it out using change of variables, using spherical coordinates:
  \begin{equation}
    \int_{0}^{2\pi}\int_{0}^{\pi}\int_0^1r^2\sin\theta drd\theta d\phi = \int_0^1r^2dr\int_0^\pi\sin\theta d\theta\int_0^{2\pi}d\phi = \frac{1}{3} \cdot 2 \cdot 2\pi = \frac{4\pi}{3}
  \end{equation}
  Of course, the volume of the sphere of radius \(r\) can be treated as a special case of the ellipsoid where \(a = b = c = r\), giving volume \(\frac{4}{3}\pi r^3\).

  \item What's the simplest kind of change of variable, save the identity? Of course, a linear change of variable. Consider a linear transformation \(T: \reals^n \to \reals^n\) and suppose \(A \subset \reals^n\) is Jordan measurable and \(B = T(A)\). Show that \(B\) is Jordan measurable and \(V(B) = V(A)|\det T|\). We split this into two cases:
  \begin{enumerate}

    \item Assume \(T\) is not invertible, i.e. \(\det T = 0\). Then the image \(I\) of \(T\) is a linear subspace of \(\reals^n\) of dimension less than \(N\), and \(B \subset I\). So all of \(B\) has \(n\)-dimensional measure zero, and hence \(V(B) = 0 = V(A)|\det T|\) as desired.

    \item Assume \(T\) is invertible, i.e. \(\det T \neq 0\). Then \(B\) is Jordan measurable and now by the change of variable theorem
    \begin{equation}
      V(B) = \int_B1 = \int_A1 \cdot |\det T| = |\det T|\int_A1 = V(A)|\det T|
    \end{equation}
    as desired.

  \end{enumerate}

\end{enumerate}
We're now going to prove the Change of Variables theorem, and it's going to take all week. Well, all the class time this week. Let's consider just one direction:
\begin{theorem}
  Let \(A \subset \reals^n\) be open and let \(g: A \to \reals^n\) be one to one and continuously differentiable such that \(\det g'(x) \neq 0\). Then if \(f: g(A) \to \reals\) is integrable, \(f \circ g|\det g'|\) is integrable on \(A\) and
  \begin{equation}
    \int_{g(A)}f = \int_A(f \circ g)|\det g'|
  \end{equation}
\end{theorem}
Note that the converse, and hence the full theorem, follows, since
\begin{itemize}
  \item As \(g\) is invertible \(f\) is locally bounded if and only if \(f \circ g|\det g'|\) is locally bounded
  \item The discontinuities of \(f\) have measure zero if and only if the discontinuities of \(f \cdot g|\det g'|\) have measure zero, since \(g\) is differentiable, using the lemma from last time.
  \item We can use \(g^{-1}\) to go from \(g(A)\) to \(A\).
\end{itemize}
We still have to show that if \(F = f \circ g|\det g'|\) is integrable on \(A\), then \(f\) is integrable on \(g(A)\). All we have to do is apply the theorem with \(F), g^{-1}\) instead of \(f, g\). Then
\begin{equation}
  F \circ g^{-1}|\det(g^{-1})'| = f \circ \cancel{g \circ g^{-1}} \cancel{|\det g' \circ g^{-1}|}\frac{1}{\cancel{|\det g' \circ g^{-1}|}} = f
\end{equation}
as desired.

We can now begin to prove the theorem in a number of steps. We're going to start by looking at a Jordan-measurable open subset \(U\) of \(A\) with closure in \(A\), e.g. an open rectangle with closure in \(A\). The first thing that we're going to do is show that it's ``good enough'' to prove the theorem on such an open subset. How? With a partition of unity! That's going to simplify things a lot, but just one thing I want to note first is that if \(f\) is integrable on \(g(A)\), then \(f\) is integrable on \(g(U)\) and \(f \circ g|\det g'|\) is integrable on \(U\).

\end{document}
